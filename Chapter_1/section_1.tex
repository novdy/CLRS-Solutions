\section{Algorithms}
\section*{\textit{Solutions}}
\section*{\underline{Exercises}}
\subsection
{Give a real-world example that requires sorting or a real-world example that requires computing a convex hull.}
    An obvious case for sorting is any kind of data visualization. Suppose you're looking for an author's earlier 
    works, then sorting by date would be useful. The convex hull problem has less obvious uses, but topography and 
    mapping come to mind.
\subsection
{Other than speed, what other measures of efficiency might one use in a real-world
setting?}
    Space is the common alternative to speed. Suppose you have a sorting Algorithm that works in $n$ time but takes 5
    petabytes of storage. It's so incredibly inefficient that you'll be unable to run the algorithm at all unless 
    you're using possibly more than one whole storage server. The real life costs of such an algorithm are 
    prohibitive.
\subsection
{Select a data structure that you have seen previously, and discuss its strengths and
limitations.}
    I will discuss two. Arrays are perhaps the most primitive data structure and allow fast access and replacement.
    However, arrays are very costly to keep in order. If you are inserting a value into a sorted array, to keep the 
    order, you will have to perform move operations on every node with a higher or lower value than the one inserted
    depending on the original order. This starts getting costly fast. Now consider a linked list. A more complex data
    structure, but one that can handle insertions easily while maintaining order. But a linked list is slow to access
    randomly and is more complex to sort initially.
\subsection
{How are the shortest-path and traveling-salesman problems given above similar?
How are they different?}
    The traveling salesman problem is essentially performing the shortest path problem over and over again. 
    Furthermore, the order of the nodes visited is undecided; this is where the complexity for the traveling 
    salesman problem arises.
\subsection
{Come up with a real-world problem in which only the best solution will do. Then
come up with one in which a solution that is “approximately” the best is good
enough.}
    A literal search is something that must be satisfied exactly. If I'm looking for the word ``book,'' I will not be 
    happy when ``look'' turns up after I run the algorithm. On the other hand, a search engine does more and often 
    will not return with an exact match, but rather a ``close enough'' match. This is useful when you don't know 
    exactly what you're looking for, but have a general idea. Interestingly, the editor I'm using right now (Visual 
    Studio Code) uses both. CTRL-F gives me an in-document literal search and CTRL-SHIFT-P gives me a command palette
    which uses a search engine to find what commands might be useful.
\solid