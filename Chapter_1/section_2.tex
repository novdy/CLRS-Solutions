\section{Algorithms as a Technology}
\section*{\textit{Solutions}}
\section*{\underline{Exercises}}
\subsection
{Give an example of an application that requires algorithmic content at the application
level, and discuss the function of the algorithms involved.}
If you haven't learned about networking layers yet, or if you need a refresher, refer to 
\url{https://en.wikipedia.org/wiki/Application_layer}. A clear example would be a security algorithm that will deny 
access to certain parts of a website if your credentials don't match with a role that would have those options.
\subsection
{Suppose we are comparing implementations of insertion sort and merge sort on the
same machine. For inputs of size $n$, insertion sort runs in $8n^2$ steps, while merge
sort runs in $64n\lg{n}$ steps. For which values of $n$ does insertion sort beat merge
sort?}
Assume $n \in \ints^+$. We are trying to find when insertion sort uses less steps than merge sort. So we are solving the 
inequality $8n^2 < 64n\lg{n}$. We are allowed to divide by $n$ since $n>0$. Along with some manipulation, we get 
$n < \lg{n^8}$. Since the inverse function of $\lg n$ is monotone increasing, we will apply it. Thus $2^n < n^8$.
Furthermore, $n^8 > 0\ \forall\ n$ and so $\dfrac{2^n}{n^8} < 1$. Clearly, $n = 1$ is not a solution, but $n=2$. Now we
must find where the upper limit is. We know there is an upper limit because $2^n$ increases faster than $n^8$. Using
a calculator (or writing a simple bit of code) will tell us that $\dfrac{2^n}{n^8} \geq 1$ when $n \geq 44$. So we have
our answer as $\boxed{1 < n < 44, n \in \ints^+.}$ You might ask why I didn't just use a calculator in the first 
place, and that's becasue computers can compute exponents signficantly faster than logarithms. It's more efficient to 
write algortithms such that they are both easier to program and faster to run.
\subsection
{What is the smallest value of $n$ such that an algorithm whose running time is $100n^2$
runs faster than an algorithm whose running time is $2^n$ on the same machine?}
Assume $n \in \ints^+$. We are solving the inequality $100n^2 < 2^n$. So $\dfrac{100n^2}{2^n} < 1$ since $2^n > 0$. 
Using a calculator or a simple program, we find that $\boxed{n > 14.}$
\solid
\section*{\underline{Problems}}
\solid
1-1 \quad For each function $f(n)$ and time $t$ in the following table, determine the largest
size $n$ of a problem that can be solved in time $t$, assuming that the algorithm to
solve the problem takes $f(n)$ microseconds.

\medskip

\underline{\textit{Answer:}}

\medskip

\begin{center}
    \begin{tabular}{c | c | c | c | c | c | c | c |}
        & 1 & 1 & 1 & 1 & 1 & 1 & 1 \\
        & second & minute & hour & day & month & year & century \\\hline \rule{0pt}{2.8ex}
        $\lg{n}$ & $2^{10^6}$ & $2^{6 \cdot 10^7}$ & $2^{3.6 \cdot 10^9}$ & $2^{8.64 \cdot 10^{10}}$ & $2^{2.592 \cdot 10^{12}}$ & $2^{3.1536 \cdot 10^{13}}$ & $2^{3.1536 \cdot 10^{15}}$  \\\hline \rule{0pt}{2.5ex}
        $\sqrt{n}$ & $10^{12}$ & $3.6 \cdot 10^{15}$ & $1.30 \cdot 10^{19}$ & $7.46 \cdot 10^{21}$ & $6.72 \cdot 10^{24}$ & $9.95 \cdot 10^{26}$ & $9.95 \cdot 10^{30}$ \\\hline \rule{0pt}{2.5ex}
        $n$ & $10^6$ & $6 \cdot 10^7$ & $3.6 \cdot 10^9$ & $8.64 \cdot 10^{10}$ & $2.59 \cdot 10^{12}$ & $3.15 \cdot 10^{13}$ & $3.15 \cdot 10^{15}$ \\\hline \rule{0pt}{2.5ex}
        $n\lg{n}$ & & & & & & & \\\hline \rule{0pt}{2.5ex}
        $n^2$ & $1000$ & $7746$ & $60000$ & $293939$ & $1609969$ & $5615693$ & $56156923$ \\\hline \rule{0pt}{2.5ex}
        $n^3$ & $100$ & $392$ & $1533$ & $4421$ & $13737$ & $31594$ & $146646$ \\\hline \rule{0pt}{2.5ex}
        $2^n$ & $6$ & $26$ & $32$ & $37$ & $42$ & $45$ & $52$ \\\hline \rule{0pt}{2.5ex}
        $n!$ &  &  &  &  &  &  &  \\\hline 
    \end{tabular}
\end{center}

Unfortunately, these numbers are so large that the answers for the functions $\sqrt{n}$, $n$, and $n\lg{n}$ must be 
rounded. Remember that these must be integer solutions if you're required to write them out. But this table is ``close 
enough'' to the actual answer that I don't care to provide a separate file enumerating the numbers in full.

In general, this problem can be stated as ``Solve for $n$ given $f(n) = t$ where $t$ is in microseconds.'' For all but
two cases my approach will be applying the inverse function to both sides for an easy solution (given one has a 
calculator). The factorial function along with $n\lg{n}$ are the only functions here that don't have a trivial inverse. 
I will provide a solution in C++ for these problems. Keep in mind that 1 second = $10^6$ microseconds. 1 minute = 60 
seconds or $(1000000)(60)$ microseconds -- and so on. In the ``1 month'' column, I will assume 1 month = 30 days. The 
longest time is going to be $(1000000)(60)(60)(24)(365)(100)$ (or 3,153,600,000,000,000) microseconds. This number is 
too large for 32 bit machines (which someone might still have), so the solution will have to be a little creative.
\solid